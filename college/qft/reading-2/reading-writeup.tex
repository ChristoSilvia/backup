\documentclass{article}

\usepackage{amsmath}

\begin{document}


\paragraph{a)}
Explain Schwartz's stragtegy for quantizing the $\phi$ field.
Specifically why does the harmonic oscillator come up, and why does it make 
	sense to express non-interactiong quantum fields and their Hamiltonians
	in terms of harmonic oscillator creation and annhilation operators 
	$a_p^{\dagger}$ and $a_p$?


\paragraph{}
Schwartz first defines the $\phi$ field classically: according to him, it staisfies 
	the simplest possible lorentz-invariant equation of motion:

\[ \partial_\mu \partial^\mu \phi = 0 \]

Schwartz then breaks this expression down into 
	$ \left( \partial_t^2 - \vec{\nabla}^2 \right) \phi = 0$
	(using the particle physics sign convention).
He then considers solutions of the form:

\[ \phi( \vec{x}, t ) = a_p(t) e^{i \vec{p} \cdot \vec{x}} \]

	and asks what the dynamics of $a_p(t)$ must be.

We can derive these dynamics:

\begin{align*}
\left( \partial_t^2 - \vec{\nabla}^2 \right) \phi & = 0\\
\left( \ddot{a_p}(t) + \vec{p} \cdot \vec{p} \right) e^{i \vec{p} \cdot \vec{x}} & = 0\\
\ddot{a_p}(t) & = - p^2 a_p(t) \\
\end{align*}

Note that the differential equation describing the dynamics of $a_p(t)$ is 
	identical to the differential equation describing a harmonic oscillator with
	angular frequency $ \left| p \right|$.
Next we will try to quantize these hamiltonians.

In order to quantize a harmonic oscillator hamiltonian with frequency $\omega$, we can 
	construct creation and annihilation operators $a$ and $a^\dagger$ 
	which obey the commutation relation $[a, a^\dagger] = 1$.
We can then express the hamiltonian as:

\[ H = \omega \left( a^\dagger a + \frac{1}{2} \right) \]

Suppose we want to associate a hamiltonian with each of the pure-momentum plane wave
	solutions to the wave equation.
We have already seen that the amplitude coefficients of a plane wave with momentum $\vec{p}$
	behave as if they are a harmonic oscillator of angular frequency $\left| \vec{p} \right|$.
All we have left to do is to define creation and annihilation operators.
If we have these creation and annihilation operators already, the dynamics of a plane wave
	of momentum $\vec{p}$ are given by:

\[ H_p = \omega_p \left( a_p^\dagger a_p + \frac{1}{2} \right) \]

To get the full hamiltonian, we can integrate over all possible values of $\vec{p}$.

\[ H_0 = \int d^p \omega_p \left( a_p^\dagger a_p + \frac{1}{2} \right) \]


\paragraph{b}
Why are there two terms in the Fourier transform of $\phi(x)$ in Eq 2.78?
Why is there a factor of $1/\sqrt{2 \omega_p}$?

\paragraph{c}
The state $|0>$


\end{document}

