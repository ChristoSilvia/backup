\documentclass{article}

\begin{document}

\section{Creation and annihilation operators}

We saw that a scalar field coupled to a spin-1 field with a gauge symmetry
	must be complex, because it needs to somehow change with that gauge
	symmetry in order for there to be any interactions.
So, we need the field to be complex.
Therefore the field needs to be different from its conjugate, and the creation
	and annihilation operators in the two fields, $\phi$ and $\phi^*$, cannot
	be the same.

The fields in 9.8 and 9.9 need to be conjugates of each other, and for the non-conjugate
	field, by convention, the creation operator is daggered and the annihilation
	operator isn't.
This convention forces $b_p^\dagger$ to be daggered.

\section{Key Insights}

\begin{enumerate}

\item
In the discussion after (9.40), why are the amplitudes only allowed to depend
	on the momenta of the photons and not their polarizations?

\item
Since amplitudes have to be lorentz-covariant, they must necessarily be 
	tensors.
Arbitrary amplitudes can be represented as lorentz tensors when proving
	statements about qft.
This is from the discussion of gauge invariance.

\item
Why is $\Pi_{\mu \alpha}$ replaced by $\xi q_\alpha q_\mu$ and not
	$\Pi_{\mu \alpha} - \xi \frac{q_\mu q_\alpha}{p^4 + i \epsilon}$?
I get this from the expression for the photon progagtor, (9.33):

\[ i \Pi_{\mu \nu} = \frac{- i \left[ g_{\mu \nu} - (1 - \xi) \frac{p_\mu p_\nu}{p^2} \right]}
	{p^2 + i \epsilon} \]

I think they only put in the change, and show that it vanishes,
	rather than putting in the full new propagator under the gauge transformation
	and showing that it's the same.
Other than that, I'm surprised that schwartz omits the factor of $p^4$ in the denominator
	of the amplitude.
Since the factor doesn't make a difference to the integral, and the integrand
	ends up vanishing, it doesn't make a difference for the proof.

\item
I take away that we need to do more work than this to show things ``in general''
	in perturbation theory.

\item
There is no 4-point vertex in QED.  
This makes me somewhat sad.
I like the 4-point vertex.

\end{enumerate}

\section{More Discussion}

Near the end of section 9.5, Schwartz discusses how conservation of 
	charge is implied by massless spin-1 particles.
He discusses how the approach, which only requires that there be an interaction,
	implies conservation of charge by itself.
It derives it with the feynmann rules.
I think this statement means, ``if we assume the feynmann rules and nothing
	else, we can derive conservation of charge''?
I eventually figured this out.

Schwartz also doesn't mention where the extra constraint which leads to eq. 9.65
	comes from.  

Schwartz also glosses over why massless particles of spin higher
	than two aren't interacting.
I think Schwartz only shows that massless particles of spin higher than
	two cannot interact with scalar fields.
He hasn't shown, and I don't know how we would show, that massless particles
	of spin higher than two cannot interact with particles of nonzero spin.

\end{document}
