\documentclass{article}

\usepackage{amssymb}
\usepackage{amsmath}

\newcommand{\Lagr}[0]{\mathcal{L}}
\newcommand{\order}[1]{\mathcal{O}\left( #1 \right)}
\newcommand{\J}[0]{\emph{J}}

\begin{document}

\section{$\hbar = c = 1$}

\section{Schwartz Problem 2.3}

The GZK bound.
In 1966 Greisen, Zarsepin, and Kuzmin argued that we should not see cosmic
	rays psigh energy protons) above a certain frequency because
	of interactions with the cosmic microwave background.

\paragraph{(a)}
The universe is a blackbody at 2.73 K.
What is the mean energy of the photons in outer space?

In a blackbody cavity, the expected energy in each mode $\omega_n$ is:

\[ \left< E_n \right> = \frac{ \omega_n }{e^{\omega_n \beta} - 1} \]

If we regard the universe as a blackbody cavity with infinte size,
	the discrete $\omega_n$'s become a continuous variable,
	and the average total energy for a given frequency becomes:

\[ E(\omega) = \frac{ \omega }{ e^{\omega \beta} - 1} \] 

Let's consider the case of a black body cube with side lengths $L$.
If we integrate through momentum-space all values of frequency,
	the average total energy becomes:

\[ E_tot = \frac{L^3}{2 \pi^2} \int_0^\infty d\omega \frac{ \omega^3 }{e^{\omega \beta} - 1} \]

If we divide by volume, we obtain an energy density:

\[ d^3 E = \frac{1}{2 \pi^2} \int_0^\infty d\omega \frac{ \omega^3 }{e^{\omega \beta} - 1} \]

\section{Non-Relativistic Classical Field Theory}

Consider the following non-relativistice Lagrangian density:

\[ \Lagr = \psi^* \left( i \partial_t + \nabla^2/2m \right) \psi - \frac{1}{2} (\psi^* \psi)^2 \]

where $\psi$ is a complex scalar field.

\subsection{a}
Consider varying the fields in $\Lagr$ starting from some arbitrary $\psi$.
Show that the resulting change in $\Lagr$ can be written in the form:

\[ \delta \Lagr = \delta \psi A + \delta \psi^* B 
	+ i ( \partial_t \delta \rho + \nabla \cdot \delta \J)
	+ \order{ \delta \psi^2 } \]

Give expressions for $A$, $B$, $\delta \rho$, and $\delta \J$ in terms of $\psi$,
	$\psi^*$, $\delta \psi$, $\delta \psi^*$.


\subsubsection{Transforming the Lagrangian into a more symmetric form}
Since the lagrangian density is interpreted as always being under an integral,
	we are allowed to use integration by parts directly on the lagrangian.

We can expand the laplacian:

\[ \Lagr = i \psi^* \partial_t \psi
	+ \psi^* \frac{\partial_x^2 +\partial_y^2 + \partial_z^2}{2m} \psi 
	- \frac{1}{2} (\psi^* \psi)^2 \]

And use the integration by parts rule to turn $\psi^* \partial_i^2 \psi$ into
	$- (\partial_i \psi^*) (\partial_i \psi)$.

\[ \Lagr = i \psi^* \partial_t \psi
	- \frac{ (\partial_x \psi^*) (\partial_x \psi)
		+  (\partial_y \psi^*) (\partial_y \psi)
		+  (\partial_z \psi^*) (\partial_z \psi)}{2m}
	- \frac{1}{2} (\psi^* \psi)^2 \]

And finally rewrite with gradient notation:

\begin{align}
 \Lagr & = i \psi^* \partial_t \psi
	- ( \nabla \psi^* ) \cdot (\nabla \psi) / 2 m
	- \frac{1}{2} (\psi^* \psi)^2
\end{align}

\subsubsection{Varying the new Lagrangian}

To the new gradient lagrangian, we can make the substitutions $\psi \to \psi + \delta \psi$
	and evaluate.

\begin{align*}
\delta \Lagr & = i (\psi^* + \delta \psi^*) \partial_t (\psi + \delta \psi)
	- ( \nabla  (\psi^* + \delta \psi^*)  ) \cdot (\nabla (\psi + \delta \psi)) / 2 m
	- \frac{1}{2} (((\psi^* + \delta \psi^*) ( \psi + \delta \psi))^2 - \Lagr \\
	& = i \psi^* \partial_t \delta \psi
	+ i \delta \psi^* \partial_t \psi
	- \nabla \psi^* \cdot \nabla \delta \psi / 2m
	- \nabla \delta \psi^* \cdot \nabla \psi / 2m
	+ \order{\delta \psi^2} \\
	& = - i \partial_t \psi^* \delta \psi
	+ i \delta \psi^* \partial_t \psi
	+ \nabla^2 \psi^* \delta \psi / 2m
	+ \delta \psi^* \nabla^2 \psi / 2m
	+ \order{\delta \psi^2} \\
\delta \Lagr & = \delta \psi \left(
		- i \partial_t \psi^*
		+ \nabla^2 \psi / 2m \right)
	+ \delta \psi^* \left(
		i \partial_t \psi
		+ \nabla^2 \psi / 2m \right)
	+ \order{\delta \psi^2}
\end{align*}

Since the lagrangian is underneath an integral, and we make the assumption that boundary
	terms are zero, we can make the following substitution freely (Schwartz 3.13):

\[ \Lagr \to \Lagr + \partial_\mu \left[ \frac{\partial \Lagr}{\partial(\partial_\mu \psi)}
	\delta \psi \right] \]

We can split this up into spatial and time terms (remember this theory is non-relativistic).
\begin{align}
\frac{\partial \Lagr}{\partial(\partial_t \psi)}
	& = i \psi^* \\
\frac{\partial \Lagr}{\partial(\nabla \psi)}
	& = - \nabla \psi^* / 2 m 
\end{align}

Therefore, for this lagrangian the following substitution will be valid:

\[ \Lagr \to \Lagr + \partial_t (i \psi^* \delta \psi) 
	+ \nabla \cdot ( - \nabla \psi^* \delta \psi / 2 m) \]

We can conclude that the total variation in the lagrangian can now be written:

\[ \delta \Lagr = \delta \psi \left(
		- i \partial_t \psi^*
		+ \nabla^2 \psi^* / 2m \right)
	+ \delta \psi^* \left(
		i \partial_t \psi
		+ \nabla^2 \psi / 2m \right)
	+ i \left( \partial_t \left( \psi^* \right) 
		+ \nabla \cdot \left( - \nabla \psi^* / 2 m \right) \right)
	+ \order{\delta \psi^2} \]

which is the form that the problem asked us to put the variation into.

If we define:
\begin{align*}
A & = - i \partial_t \psi^* + \nabla^2 \psi^* / 2m \\
B & = - i \partial_t \psi + \nabla \psi^* / 2m\\
\rho & = \psi^* \delta \psi\\
\J & = - \nabla \psi^* \delta \psi / 2 m 
\end{align*}

then the above lagrangian is:

\[ \delta \Lagr = \delta \psi A + \delta \psi^* B 
	+ i ( \partial_t \delta \rho + \nabla \cdot \delta \J)
	+ \order{ \delta \psi^2 } \]

as desired.

\subsection{Classical Equations of Motion}

The classical equations of motion for the theory:

\[ \Lagr  = i \psi^* \partial_t \psi
- ( \nabla \psi^* ) \cdot (\nabla \psi) / 2 m
- \frac{1}{2} (\psi^* \psi)^2 \]

are given by the euler lagrange equations:

\begin{align*}
\frac{\partial \Lagr}{\partial \psi} 
	- \partial_\mu \left[ \frac{\partial \Lagr}{\partial(\partial_\mu \psi)} \right]
	& = 0\\
\frac{\partial \Lagr}{\partial \psi^*} 
	- \partial_\mu \left[ \frac{\partial \Lagr}{\partial(\partial_\mu \psi^*)} \right]
	& = 0
\end{align*}

We can compute the derivatives with respect to the lagrangian:

\begin{align*}
\frac{\partial \Lagr}{\partial \psi} & = - (\psi^*)^2 \psi\\
\frac{\partial \Lagr}{\partial \psi^*} & = i \partial_t \psi - (\psi)^2 \psi^*\\
\frac{\partial \Lagr}{\partial(\partial_t \psi)} & = i \psi^*\\
\frac{\partial \Lagr}{\partial(\partial_t \psi^*)} & = 0\\
\frac{\partial \Lagr}{\partial(\nabla \psi)} & = - \nabla \psi^* / 2m \\
\frac{\partial \Lagr}{\partial(\nabla \psi^*)} & = - \nabla \psi / 2m \\
\end{align*}

We can substitute these derivatives into the euler-lagrange equations to obtain:

\begin{align*}
i \partial_t \psi & = - \nabla \psi / 2 m + G \psi^* \psi^2 \\
- i \partial_t \psi^* & = - \nabla \psi^* / 2 m + G (\psi^*)^2 \psi \\
\end{align*}

\subsection{Conserved Current}

The lagrangian density is unchanged by the transformation:

\begin{align*}
\psi \to e^{i \epsilon} \psi & & \psi^* \to e^{- i \epsilon} \psi^* 
\end{align*}

for any real value of $\epsilon$.
What is the conserved current corresponding to this symmetry?
What is the corresponding conserved quantity?

\paragraph{}
Since the lagrangian is unchanged under the above symmetry, we conclude that:

\[ \frac{\delta \Lagr}{\delta \epsilon} = 0 \]

We can compute $\frac{\delta \Lagr}{\delta \epsilon}$, to get (Schwartz 3.22):

\begin{align*}
 \frac{\delta \Lagr}{\delta \epsilon} 
	 & = \left[ \frac{\partial \Lagr}{\partial \psi} - \partial_\mu
		\frac{\partial \Lagr}{\partial(\partial_\mu \psi)} \right]
	\frac{\delta \psi}{\delta \epsilon}
	+ \partial_\mu \left[ \frac{\partial \Lagr}{\partial( \partial_\mu \psi)}
	\frac{\delta \psi}{\delta \epsilon} \right]\\
& + \left[ \frac{\partial \Lagr}{\partial \psi^*} - \partial_\mu
		\frac{\partial \Lagr}{\partial(\partial_\mu \psi^*)} \right]
	\frac{\delta \psi^*}{\delta \epsilon}
	+ \partial_\mu \left[ \frac{\partial \Lagr}{\partial( \partial_\mu \psi^*)}
	\frac{\delta \psi^*}{\delta \epsilon} \right]
\end{align*}

If the equations of motion are satisfied, and the symmetry is really a symmetry (which it is),
	the whole expression reduces into the form:

\[ \partial_\mu \left[ \frac{\partial \Lagr}{\partial( \partial_\mu \psi)}
	\frac{\delta \psi}{\delta \epsilon} 
	+ \frac{\partial \Lagr}{\partial( \partial_\mu \psi^*)}
	\frac{\delta \psi^*}{\delta \epsilon} \right] = 0 \]

Let's compute the variations under the symmetry:

\begin{align*}
\frac{\delta \psi}{\delta \epsilon} & = i \psi\\ 
\frac{\delta \psi^*}{\delta \epsilon} & = -i \psi^*\\
\end{align*}

There is a conserved 4-vector associated with the equations:

\begin{align*}
p & = - \psi^* \psi \\
\vec{J} & = - i \nabla \psi^* \psi/2m + i \nabla \psi \psi^* / 2m 
\end{align*}

The conserved quantity is:

\[ \int d^x p = \int d^x - \psi^* \psi \]

This has an interesting interpretation that if $\psi$ is treated as a wave function
	as in quantum mechanics, the normalization is conserved.
We should think about the conserved current therefore as a probability current.

\subsection{d}

Does this have a time-reversal symmetry?

Suppose $\psi(x, t)$ is a solution.
Consider $\psi^*(x, -t)$.

\begin{align*}
i \partial_t ( \psi^*(x, -t)) & = - \nabla \psi^*(x, -t)/2m + G \psi(x, -t)(\psi^*(x, -t))^2 \\
- i \partial_t \psi^*(x, -t) & = - \nabla \psi^*(x, -t)/2m + G \psi(x, -t)(\psi^*(x, -t))^2
\end{align*}

Which is the other equation of motion which is satisfied.
Therefore, if $\psi(x,t)$ is a solution, $\psi^*(x, -t)$ is also a solution.

\end{document}
