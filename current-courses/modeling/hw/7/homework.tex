\documentclass{article}

\usepackage{tikz}
\usepackage{pgfplots}
\usepackage{amsmath}
\usepackage{amsfonts}
\usepackage[utf8]{inputenc}
\usepackage{graphicx}
\usepackage{fullpage}

\pgfplotsset{compat=1.11}
\pgfplotstableread[ col sep = comma]{pareto-front-2.csv}\pareto

\begin{document}

\begin{flushright}
Chris Silvia

Problem Set 8

(with a little help from Sascha Hernández)

\end{flushright}

\section{Fish Tank Simulation Problem}

\subsection{Problem Statement}

You are a mathematical modeler hired by a store owner, who owns
	and operates a store which sells fish tanks, as well as other
	merchandise.
Customers who wish to buy fish tanks are relatively rare.
In the store owner's experience, they arrive about once per week.
Fish tanks need to be ordered from a manufacturer; the tank takes
	several days to arrive after it is ordered.
The store owner wants to know the best scheme for ordering fish
	tanks.
She has thought up two different schemes for ordering the fish tanks:

\begin{enumerate}
\item Order a new tank each time one is sold.
This way, there will never be more than one in stock at any given time,
	so there will be no waste of shelf space.
But, the store may also be caught without any fish tanks, and lose
	the customer.
\item Order a new tank about once per week.
We will avoid having more than necessary in stock, but they may 
	accumulate.
\end{enumerate}

The store owner wants me to determine which strategy is better,
	and if there is a still-better strategy.

\subsection{Determination of Parameters} \label{sec:det-of-params}

After consulting with the client, I determined some parameters:

\paragraph{}
\begin{tabular}{l|l|l}
Possibility of Customer per Day & $a$ &  1/7 \\
Length of Fishtank Ordering Process & $O$ & 5 days \\
Profit from a Sale & $P$ & \$20 \\
Loss from Customer Not Finding Tank & $L_c$ & \$10 \\
Loss per Additional Fish Tank per Day & $L_t$ & \$0.5 / day \\
\end{tabular}

\paragraph{}
In a single day, if a customer comes to the store, and there are
	$m$ fishtanks at the beginning of the day, the total profit is:

\begin{align}
	P - (m - 1) L_t & \text{    if } m \geq 1 \nonumber \\
	- L_c & \text{    if } m = 0 
\end{align}

If no customers come to the store, then the total profit is:

\begin{align}
	- m L_t
\end{align}

In this model, each day there is a probability $a$ of one customer 
	arriving, and a probability $1 - a$ of no customers arriving.
This model doesn't ever allow two customers in the store per day.
A more sophisticated model may take this into account.

\subsection{Description of the Model}

This is a monte carlo model.
It consists of a program which uses random numbers to model
	the randomness observed in real life.
To make predictions, I will use a large number of program evaulations
	as data, and then use statistical techniques to analyze the data.
I will be most interested in the mean and variance of the
	final profit.
I will present a pareto front of in mean-variance space to the
	client, and she will chose the ideal point.

In the program, time will be discretized in days.
The program will compute a large number of days.
Each day, a single customer will arrive with probability $a$.
The profits will be computed as described in section
	(\ref{sec:det-of-params}), and the running total profit will be
	computed.
Each day, if no fish tank is on order, our strategy will decide whether 
	or not to order a fish tank.
Our model will only allow us to order one fish tank at any given time,
	so we cannot order several at once.

From each run of this program, we will record the expected profit
	as data.
This will be the data which we later compute the mean and standard
	deviation of.

\subsection{Additional Strategies}

I can add some additional strategies:

\begin{enumerate}
\setcounter{enumi}{2}
\item Order a tank every $i$ days, and also order a new tank every 
	time the tanks run out.
\item Stop ordering tanks once you have an inventory of $M$ tanks.
\end{enumerate}

\subsection{Implementing the Strategies}

If I order a tank every $i$ days, and stop when I have an inventory
	of $M$ tanks, then strategy 1 is the case of $i = \infty$
	and $M = \infty$ and a new tank is ordered every time the tanks run out,
	strategy 2 is the case $M = \infty$ where no new tank is ordered
	every time the tanks run out,
	strategy 3 is the case $M = \infty$ where a new tank is ordered
	every time tanks run out, and strategy 4 can be either when a tank
	is ordered or not ordered every time tanks run out.
The choice of ordering or not ordering can be recorded as a boolean 
	variable $b$.
Together, these strategies are summarized by three parameters:

\begin{align*}
i \in \mathbb{Z}_+ \cup \{ \infty \} \\
M \in \mathbb{Z}_+ \cup \{ \infty \} \\
b \in \{ 0, 1 \} 
\end{align*}

Determining the optimal value of the parameters will determine the optimal
	strategy among the listed strategies.
I have no way of guarenteeing that the optimal parameters I find
	yield an optimal strategy.

\subsection{Tests of the Strategies}

I may want to minimize the expected standard deviation, or maximize the
	expected profit.
I tested strategies with maximum inventory ranging from 1 to 20, and with 
	the maximum inventory set to $\infty$, and with the re-ordering inverval
	ranging from 1 to 30 days.
I tested each strategy with re-ordering upon purchase, and without.
For each strategy I computed 200 iterations.

Given these conditions, I present a pareto front of points:

\begin{tikzpicture}
\begin{axis}[
	xlabel = {Expected Profit},
	ylabel = {Standard Deviation of Profit}
]
\addplot[only marks] table[x index = {3}, y index = {4}]{\pareto};
\end{axis}
\end{tikzpicture}

\paragraph{}
\begin{tabular}{l|l|l|l|l}
Order when sold out? & Max Inventory & Order Schedule & Expected Profit & Std Div. of Profit \\
\hline
no&1&2&1805.2675&189.6323000966601\\
no&1&4&1719.635&176.5253246028191\\
no&1&5&1647.86&140.2045988822717\\
no&1&6&1493.0425&131.8275367324853\\
no&1&7&1357.5675&116.5302889163355\\
no&1&8&1213.07&94.8633113442493\\
no&1&9&1068.945&79.78480463253172\\
no&2&8&1262.9625&108.388445506644\\
no&2&9&1114.5175&89.76312179241314\\
no&2&10&984.145&76.01933088591156\\
no&3&9&1117.855&93.37847273785468\\
no&3&11&845.1975&69.92698808085689\\
no&6&11&852.9275&75.59357071482391\\
\end{tabular}
\paragraph{}

None of the pareto-optimal strategies recommend ordering when sold out.
I am surprised by this, it felt like a good idea.
The maximum inventory is only ever 6, which makes me think that establishing
	a maximum inventory is a good idea.
The order schedules are larger with larger inventory limits, to account for them.

There is a range of low-value low-variance strategies and high-variance 
	high-value strategies.

Attached are histograms which verify the gaussian-ness of these points
	with histograms of 2000 samples.

\section{Making Events Independent}

Consider a model of a queue, which has three states:

\begin{enumerate}
\item No customers are present
\item A customer is being served
\item A customer is being served, and another is waiting
\end{enumerate}

This assumes that the line can never grow longer than one, or that
	nobody will join a line when it already has one person in it.

The model also assumes that a customer arrives with probability $p$,
	and a customer currently being served finishes with probability $q$.

In state 1 there is no customer currently being served to finish,
	and in state 3 there is no customer to arrive.
However, in state 2, both events could potentially happen.

\subsection{Characteristics of Independent Events}

Suppose event A happens with probability $p$, and event B happens with
	probability $q$.
If A and B are independent, then if A happens, then the probability of 
	B happening is still $q$.
If A does not happen, then the probability of $B$ happening is also still $q$.

\subsection{Modification of Transition Probabilities}

Suppose a customer is being served.
We have assumed that a probability of a customer arriving in a given timestep is  $p$, 
	and the probability of a customer currently being served finishing is $q$.
Therefore the following events could happen:

\paragraph{}
\begin{tabular}{l | l | l | l }
Customer Arrives & Current Customer Finishes & Probability & Result State\\
\hline
false & false & $(1 - p) (1 - q)$ & State 2\\
false & true  & $(1 - p) q $ & State 1\\
true & false & $p (1 - q)$ & State 3\\
true & true & $p q$ & State 2 \\
\end{tabular}
\paragraph{}

The old markov matrix is:

\begin{align}
M & = \left( \begin{matrix}
1-p & p & 0 \\
q & 1 - (q + p) & p \\
0 & q & 1 - q 
\end{matrix} \right)
\end{align}

The new markov matrix is:

\begin{align}
M & = \left( \begin{matrix}
	1 - p & p & 0 \\
	(1 - p)q & (1 - p)(1 - q) + p q & p(1 - q) \\
	0 & q & 1 - q
\end{matrix} \right) \nonumber \\
  & = \left( \begin{matrix}
	1 - p & p & 0 \\
	(1 - p)q & 1 - (p + q) + 2 p q & p(1 - q) \\
	0 & q & 1 - q
\end{matrix} \right)
\end{align}

The modification is in the second line.
The first entry changes from $q$ to $(1 - p)q$, which are very similar if $p$ is small.
The second entry changes $1 - (p + q)$ to $1 - (p + q) + 2 p q$, which are also very 
	similar of $p$ and $q$ are small.
Note that if $p$ and $q$ are not small, the old transition probability isn't
	 guarenteed to be positive!
Finally, the third entry changes from $p$ to $p(1 - q)$, which is very similar
	if $q$ is small.

\subsection{Result in the case $p = 0.2$, $q = 0.2$}

For the case $p = 0.2$, $q = 0.2$, the corresponding matrices' eigenvectors and eigenvalues
	are, for the old probabilities:

\begin{align*}
\lambda_1 & = 0.4 & v_1 & = \left[ 
		\begin{matrix} 3.6772 \times 10^{15} \\ -7.3543 \times 10^{15} \\3.6673 \times 10^{15} 
	\end{matrix} \right] \\
\lambda_2 & = 0.8 & v_2 & = \left[
		\begin{matrix} -1.5923 \times 10^{15} \\ 0.46875 \\ 1.5923 \times 10^{15}
	\end{matrix} \right] \\
\lambda_3 & = 1.0 & v_3 & = \left[
		\begin{matrix} 1/3 \\ 1/3 \\ 1/3
	\end{matrix} \right]
\end{align*}

and for the new probabilities:

\begin{align*}
\lambda_1 & = 0.4 & v_1 & = \left[ 
		\begin{matrix} 2.5 \\ -4.0 \\ 2.5  
	\end{matrix} \right] \\
\lambda_2 & = 0.8 & v_2 & = \left[
		\begin{matrix} 1.5923 \times 10^{15} \\ 0.33413 \\ - 1.5923 \times 10^{15}
	\end{matrix} \right] \\
\lambda_3 & = 1.0 & v_3 & = \left[
		\begin{matrix} 1/3 \\ 1/3 \\ 1/3
	\end{matrix} \right]
\end{align*}

Since the probability after a long run of trials is the eigenvector corresponding
	to the eigenvalue 1, in both cases the states are equally likely. 

\end{document}
