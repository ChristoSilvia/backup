\documentclass{article}

\usepackage{amsmath}
\usepackage{amsfonts}
\usepackage{graphicx}

\DeclareMathOperator\erf{erf}

\begin{document}

%	\begin{flushright}
%	Chris Silvia
%	(with help from Sascha Hernandez)
%	\date
%	\end{flushright}

\section{Lottery}

Suppose each week, $2000000$ lottery tickets are sold for \$1 apiece.
If $4000$ of these tickets pay off \$30 each,
	$500$ pay off \$800 each, one ticket pays off \$1200000,
	and no ticket pays off more than one prize?

\subsection{Expected Value of one ticket}

Since no ticket pays off more than one prize, and each ticket is only
	one option, the expected value of a ticket is:

\begin{align*}
E\left( \text{Ticket} \right)
	& = \frac{
		N_\text{no-prize tickets} 
			0
		+ N_\text{\$30-prize tickets}
			30
		+ N_\text{\$800-prize tickets} 800
		+ N_\text{\$1200000-prize tickets} 1200000}{N_\text{tickets}} \\
	& = \frac{0 + 120000 + 400000 + 1200000}{1995499}\\
E\left( \text{Ticket} \right)
	& = 0.86
\end{align*}

This lottery seems like a losing proposition.

\subsection{Expected Value of Five Tickets}

One might naively conclude that the expected value of five tickets
	is $5 E\left( \text{Ticket} \right)$.
However, that is not correct.
The \$1,200,000 ticket goes into the expected value, but it can only
	be one of the tickets.
Therefore, the expected value will be less than $5 E\left( \text{Ticket} \right)$.

\subsubsection{Expected Value of Five Tickets}

To compute the expected value of five tickets, I computed the likelihood that each
	possible combination of tickets could arise, and then multiplied it	by
	the payoff of that combination.
To my \emph{complete and utter shock}, the expected value for 5 tickets which I got 
	is 4.3, exactly 5 times the expected value for one ticket by itself.
I am completely astounded and I do not believe my result. 
I have reimplemented this several times, once in haskell, and I keep getting the same
	result over and over again.
I don't believe it's true, but I'm not sure what to do at this point.

%	To prepare for this calculation, I will first calculate the expected value
%		of two tickets.
%	
%	\begin{align*}
%	E\left( \text{Two Tickets} \right)
%		& =	P\left( \text{Ticket 1 is worth \$ 0} \right)
%			\left( 0 
%				+ E\left( \text{Ticket 2} \mid \text{Ticket 1 is worth \$0} \right)\right)\\
%		&	+ P\left( \text{Ticket 1 is worth \$ 30} \right)
%			\left( \$30 
%				+ E\left( \text{Ticket 2} \mid \text{Ticket 1 is worth \$30} \right)\right)\\
%		&	+ P\left( \text{Ticket 1 is worth \$ 800} \right)
%			\left( \$800
%				+ E\left( \text{Ticket 2} \mid \text{Ticket 1 is worth \$800} \right)\right)\\
%		&	+ P\left( \text{Ticket 1 is worth \$ 1,200,000} \right)
%			\left( \$1,200,000
%				+ E\left( \text{Ticket 2} \mid \text{Ticket 1 is worth \$1,200,000} \right)
%			\right)\\
%		& = \frac{1995499}{2000000} 
%			\left( \frac{120000 + 400000 + 1200000}{1999999} \right)\\
%		& + \frac{4000}{2000000} 
%			\left( 30 + \frac{110070 + 400000 + 1200000}{1999999} \right)\\
%		& + \frac{500}{2000000} 
%			\left( 800 + \frac{120000 + 399500 + 1200000}{1999999} \right)\\
%		& + \frac{1}{2000000} 
%			\left( 1200000 + \frac{120000 + 400000}{1999999} \right)\\
%		& = 0.85806 + 0.061710 + 0.20021 + 0.6 \\
%		& = 1.71998
%	\end{align*}
%	
%	This is very slightly smaller than $2 * E\left( \text{Ticket} \right) = 1.72$,
%		as predicted.
%	
%	\subsubsection{Five Tickets}
%	
%	\paragraph{}
%	\begin{tabular}{| l | l | l | l | l |}
%	\hline
%	\# \$1,200,000 & \# \$ 800 & \# \$ 30 & \# \$ 0 & Degeneracy \\
%	\hline
%	1 & 0 & 0 & 0 & 4 & ${1995499 \choose 4}\\
%	1 & 0 & 0 & 1 & 3 & \\
%	1 & 0 & 0 & 2 & 2 & \\
%	1 & 0 & 0 & 3 & 1 & \\
%	1 & 0 & 0 & 4 & 0 & \\
%	1 & 0 & 1 & 0 & 3 & \\
%	1 & 0 & 1 & 1 & 2 & \\
%	1 & 0 & 1 & 2 & 1 & \\
%	1 & 0 & 1 & 3 & 0 & \\
%	1 & 0 & 0 & 1 & 3 & \\
%	\end{tabular}

\section{Aircraft Departure}

\subsection{Problem Statement}

To prepare an aircraft for departure, it takes a random amount of time
	between 20 and 27 minutes.
If the flight is scheduled to depart at 9:00 am and preparation begins at 8:37
	am, find the probability that the plane is prepared for departure on
	schedule.

\subsection{Solution}

Since time is a continuous variable, it makes sense to model the random
	amount of time between 20 and 27 minutes as a probability distribution
	on the interval $(20,27)$.
This does not preclude the possibility that the time is discreet, this can 
	be modeled by a series of dirac delta distributions at the integer points.
A distribution $\mu(t)$ is therefore a valid probability distribution for flight
	times if it is supported only on $(20,27)$ and is normalized.

Given $\mu(t)$, the expected time that the loading takes is:

\begin{align}
\langle t \rangle & = \int_{20}^{27} t \mu(t) dt
\end{align}

If a flight is scheduled to depart at 9:00 am and preparation begins at 8:37
	am, then the probability that the filght will be prepared for departure
	in time is the probability that the preparations take fewer than 23 minutes.
This is given by:

\begin{align}
P\left( \text{Prepared for Departure} \right)
	& = \int_{20}^{23} \mu(t) dt
\end{align}

So far I have made no assumptions.

\subsection{Example}
There is no way of determining what the probability of being prepared for depature
	is without knowing the probability distribution $\mu(t)$.
Let's evaluate the probability with the constant distribution:
\begin{align}
\mu(t) & = \begin{cases} \frac{1}{7} & 20 < t < 27 \\
	0 & \text{otherwise} \end{cases}
\end{align}

\begin{align}
P\left( \text{Prepared for Departure} \right)
	& = \int_{20}^{23} \frac{1}{7} dt \nonumber \\
	& = \frac17 \left( t \right)^{t = 23}_{t = 20} \nonumber \\
	& = \frac17 \left(23 - 20\right) \nonumber \\
	& = \frac37
\end{align}

\section{Nigerian Con}

\begin{itemize}
\item The spammer sends the initial solicitation to $N = 10000000$ recipients.
\item $x = 5\%$ are Very Gullible (VG) and the remaining $1-x = 95\%$ are Reasonably
	Skeptical (RS).
The spammer does not know which one is which.
\item The spammer's cost of follow-up email exchange is $K = \$20$.
\item If the follow-up email exchange is a success, the spammer gets 
	a reward of $L = \$2000$.
\item The spammer can send either Well-Written Solicitations (WWS)
	or Badly Written Solicitations (BWS).
\item The probability of initial response from each VG individual is $p_g = 0.01$
	regardless of the type of solicitation.
\item The probability of initial response from each RS individual is $q_w = 0.002$
	if the spammer sends WWS and $q_b = 0.001$ if the spammer sends BWS.
\item Once the recipient responds, the probability of successfully fooling
	him is $r_1 = 0.1$ for VG responders and $r_2 = 0.005$ for RS.
\end{itemize}

\subsection{Expected Cost from Follow-Up Emails}

Suppose we send well-written emails.

\begin{align}
E\left( \text{Responses} \mid \text{WWS} \right) 
	& = x N p_g + (1 - x) N q_w \nonumber \\
	& = 5000 + 19000 \nonumber \\
	& = 24000 \nonumber \\
E\left( \text{Follow-Up Cost} \mid \text{WWS} \right)
	& = K \left( x N p_g + (1 - x) N q_w \right) \nonumber \\
	& = K * 5000 + K * 19000 \nonumber \\
	& = \$ 100000 + \$ 380000\nonumber\\
	& = \$ 480000 \nonumber \\
E\left( \text{Gross Revenue} \mid \text{WWS} \right)
	& = E\left( \text{Gross Revenue from VG} \mid \text{WWS} \right)
	+ E\left( \text{Gross Revenue from RS} \mid \text{WWS} \right) \nonumber \\
	& = L \left( 
		E\left( \text{Follow-Ups from RS} \mid \text{WWS} \right)
		+ E\left( \text{Follow-Ups from VG} \mid \text{WWS} \right) \right) \nonumber \\
	& = L \left( 
		r_2 E\left( \text{Responses from RS} \mid \text{WWS} \right)
		+ r_1 E\left( \text{Responses from VG} \mid \text{WWS} \right) \right) \nonumber \\
	& = L \left( 
		r_1 p_g x N
		+ r_2 q_w (1-x) N  \right) \nonumber \\
	& = L N \left( 5 \times 10^{-5} + 9.5 \times 10^{-6} \right)
	& = \$ 1190000 \nonumber \\
E\left( \text{Profit} \mid \text{WWS} \right)
	& = E\left( \text{Gross Revenue} \mid \text{WWS} \right)
	- E\left( \text{Follow-Up Cost} \mid \text{WWS} \right) \nonumber \\
	& = N \left( L \left( 
		r_1 p_g x
		+ r_2 q_w (1-x)  \right) 
	- K \left(x p_g + (1 - x) q_w \right) \right) \nonumber\\
	& = \$ 1190000 - \$ 480000 \nonumber \\
	& = \$ 710000 \nonumber \\
\end{align}

Most of the expense comes from sending emails to the RS individuals,
	although they contribute only 15 \% of the revenue.

\begin{align}
E\left( \text{Responses} \mid \text{BWS} \right)
	& = x N p_g + (1 - x) N q_b \nonumber \\
	& = 5000 + 9500 \nonumber \\
	& = 14500 \nonumber \\
E\left( \text{Follow-Up Cost} \mid \text{BWS} \right) 
	& = K \left(x N p_g + (1 - x) N q_b \right) \nonumber \\
	& = K * 5000 + K * 9500 \nonumber \\
	& = 100000 + 190000 \nonumber\\
	& = \$ 290000 \nonumber\\ 
E\left( \text{Gross Revenue} \mid \text{BWS} \right)
	& = E\left( \text{Gross Revenue from VG} \mid \text{BWS} \right)
	+ E\left( \text{Gross Revenue from RS} \mid \text{BWS} \right) \nonumber \\
	& = L \left( 
		E\left( \text{Follow-Ups from RS} \mid \text{BWS} \right)
		+ E\left( \text{Follow-Ups from VG} \mid \text{BWS} \right) \right) \nonumber \\
	& = L \left( 
		r_2 E\left( \text{Responses from RS} \mid \text{BWS} \right)
		+ r_1 E\left( \text{Responses from VG} \mid \text{BWS} \right) \right) \nonumber \\
	& = L \left( 
		r_1 p_g x N
		+ r_2 q_w (1-x) N  \right) \nonumber \\
E\left( \text{Gross Revenue} \mid \text{BWS} \right)
	& = L N \left( r_1 p_g x + r_2 q_b (1-x) \right) \nonumber \\
	& = L N \left( 5 \times 10^{-5} + 4.75 \times 10^{-6} \right) \nonumber \\
	& = \$ 1095000 \nonumber \\
E\left( \text{Profit} \mid \text{BWS} \right)
	& = E\left( \text{Gross Revenue} \mid \text{BWS} \right)
	- E\left( \text{Follow-Up Cost} \mid \text{BWS} \right) \nonumber \\
	& = N \left( L \left( 
		r_1 p_g x
		+ r_2 q_b (1-x)  \right) 
	- K \left(x p_g + (1 - x) q_b \right) \right) \nonumber\\
	& = \$ 1095000 - \$ 290000 \nonumber \\
	& = \$ 805000 \nonumber
\end{align}

Although the expected revenue doesn't change significantly,
	the cost of sending follow-ups decreases, making this strategy
	more profitable.

\subsection{How does email cost affect strategy?}

We have determined:

\begin{align}
E\left( \text{Profit} \mid \text{WWS} \right)
	& = N \left( L \left( 
		r_1 p_g x
		+ r_2 q_w (1-x)  \right) 
	- K \left(x p_g + (1 - x) q_w \right) \right) \nonumber \\
E\left( \text{Profit} \mid \text{BWS} \right)
	& = N \left( L \left( 
		r_1 p_g x
		+ r_2 q_b (1-x)  \right) 
	- K \left(x p_g + (1 - x) q_b \right) \right) \nonumber
\end{align}

Therefore, the two strategies are equivalent if  
	$E\left( \text{Profit} \mid \text{WWS} \right) = 
	E\left( \text{Profit} \mid \text{BWS} \right)$,
or:

\begin{align}
	N \left( L \left( 
		r_1 p_g x
		+ r_2 q_w (1-x)  \right) 
	- K \left(x p_g + (1 - x) q_w \right) \right) \nonumber \\
 = N \left( L \left( 
		r_1 p_g x
		+ r_2 q_b (1-x)  \right) 
	- K \left(x p_g + (1 - x) q_b \right) \right) \nonumber \\
	L \left( 
		r_2 q_w (1-x)  \right) 
	- K (1 - x) q_w
& = L \left((1 - x) 
		r_2 q_b  \right) 
	- K \left((1 - x) q_b \right) \nonumber \\
	L r_2 q_w
	- K  q_w
& = L r_2 q_b
	- K  q_b  \nonumber \\
K (q_b - q_w) & = L r_2 (q_b - q_w) \nonumber \\
K & = L r_2\\
K & = \$ 10 
\end{align}

As long as the exchanges cost more than $\$10$, one should send badly
	written emails.
If exchanges cost fewer than $\$10$, then well-written emails make sense.

\section{Speech Problem}

\subsection{Problem Statement}

Suppose a politician is giving a speech, and wants to know how much
	emphasis to put on one of her policy positions.
We assume that the voters are split into two groups, Pro and Con, 
	on that position.
We further assume that the Candidate has supporters in both groups.
The number of people in each group is

\paragraph{}
\begin{tabular}{|l|l|l|}
\hline
Pro? & Support Candidate? & \# Individuals\\
\hline
Yes & Yes & $r$ \\
Yes & No & $n-r$ \\
No & Yes & $s$ \\
No & No & $m-s$ \\
\hline
\end{tabular}

For each mention of the issue:
\begin{itemize}
\item Each Non-Supporter who supports the issue becomes a supporter
	with probability $p$.
\item Each Supporter who doesn't support the issue becomes a non-supporter
	with probability $q$.
\end{itemize}

\subsection{How Many Times to Mention the Issue?}

Suppose the candidate mentions the issue once.
Then, she can expect to gain $ (n-r) p$ supporters,	
	and lose $ q s $ supporters.

Now, if she mentions the issue again, the number of supporters
	has changed:

\paragraph{}
\begin{tabular}{|l|l|l|}
\hline
Pro? & Support Candidate? & \# Individuals\\
\hline
Yes & Yes & $r + (n - r) p$ \\
Yes & No & $(n-r)(1-p)$ \\
No & Yes & $s(1 - q)$ \\
No & No & $m-s + q s$ \\
\hline
\end{tabular}

\paragraph{}
After mentioning the issue a second time, she can expect the numbers to be:

\paragraph{}
\begin{tabular}{|l|l|l|}
\hline
Pro? & Support Candidate? & \# Individuals\\
\hline
Yes & Yes & $r + (n - r) p + (n-r)(1-p)p$ \\
Yes & No & $(n-r)(1-p) - (n-r)(1-p)p$ \\
No & Yes & $s(1 - q) - s(1-q)q$ \\
No & No & $m-s + q s + s(1-q)q$ \\
\hline
\end{tabular}

\subsection{Generalization}

\subsubsection{Determining Net Change in supporters from $k$ mentions}

The probability of an individual being convinced after the $n$-th
	mentioning of an issue is the probability that they are not
	convinced the previous $n-1$ times times the probability that
	they are convinced the $n$-th time.
If the probabilities are the same each time, then if the probability
	of being convinced each time is $p$, the probability of being
	convinced the $n$-th time is $(1-p)^{n-1}p$.

Therefore, if there are $(n-r)$ individuals who support the candidate,
	and the candidate mentions the issue $k$ times,
	the expected number of people who are convinced is the sum of the
	expected number of people who are convinced each time.
Therefore, the expected number of people to be convinced is:

\begin{align*}
E\left( \text{\# New Supporters} \mid \text{k mentions} \right)
	& = (n - r) \left( p + (1 - p) p + (1 - p)^2 p + \dots + (1 - p)^{k-1} p \right)\\
	& = (n - r) p \sum_{i=0}^{k-1} ( 1 - p )^i \\
	& = (n - r) p \left( \frac{1 - (1 - p)^k}{1 - (1 - p)} \right) \\
E\left( \text{\# New Supporters} \mid \text{k mentions} \right)
	& = (n - r) \left( 1 - (1 - p)^k \right)\\
E\left( \text{\# Lost Supporters} \mid \text{k mentions} \right)
	& = s \left( 1 - (1 - q)^k \right)\\
\end{align*}

Therefore, if the candidate mentions the issue $k$ times, the expected
	change in number of supporters is:

\begin{align*}
E\left( \text{\# Net Change in Supporters} \mid \text{k mentions} \right)
	& =  (n - r) \left( 1 - (1 - p)^k \right)
	- s \left( 1 - (1 - q)^k \right)
\end{align*}

This makes sense: if the candidate mentions the issue $0$ times, then she
	will expect no change in supporters under this model.
If the candidate mentions the issue an infinite number of times,
	everybody who can be convinced will be, and therefore her support
	will change by $(n-r) - s$.

\subsubsection{Determining the Optimal Continuous Value for $k$}

Let's assume that we can mention the issues a fractional number of times,	
	so that we can use calculus to maximize yield.
We will revert to integers later.

\begin{align*}
\frac{\partial 
	E\left( \text{\# Net Change in Supporters} \mid \text{k mentions} \right)}
{\partial k}
	& =  (n - r) \log \left( \frac1{1-p} \right) (1 - p)^k
	- s \log \left( \frac1{1-q} \right) (1 - q)^k \\
\end{align*}

If we set 
$\frac{\partial 
	E\left( \text{\# Net Change in Supporters} \mid \text{k mentions} \right)}
{\partial k}$ = 0, then we get:

\begin{align*}
	(n - r) \log \left( \frac1{1-p} \right) (1 - p)^k
	& =  s \log \left( \frac1{1-q} \right) (1 - q)^k \\
	\log \left( (n - r) \log \left( \frac1{1-p} \right) \right)
	+ k \log \left( 1 - p \right)
	& =  
	\log \left( s \log \left( \frac1{1-q} \right) \right)
	+ k \log \left( 1 - q \right)\\
	k & = \frac{ \log \left(
		\frac s{n-r}
		\frac{ \log \left( 1 - q \right) }
			{ \log \left( 1 - p \right) } \right) } 
	{ \log \left( \frac{1 - p}{1 - q} \right)}
\end{align*}

, assuming that $p \neq q$.

Note that $k$ has to be a positive value, so therefore,
	for this unique local extremum to be useful,
	either

\begin{align*}
\frac s{n-r} \frac{ \log (1 - q)}{\log (1 - p)} & > 1\\
\frac{1 - p}{1 - q} & > 1
\end{align*}

or 

\begin{align*}
\frac s{n-r} \frac{ \log (1 - q)}{\log (1 - p)} & < 1\\
\frac{1 - p}{1 - q} & < 1
\end{align*}

If the unique local extremum exists, then if it is greater than
	the boundary values, it is the optimum point within the interval.
I could check the second derivative, but I think that it will 
	be more tractable to just compare the expected change in support at the 
	extremum of $k$ to the values at $0$ and $\infty$.

%	Since there is a unique optimal continuous value for $k$, then it is the unique
%		local extremum.
%	To determine whether it is a maximium or a minimum, we take the second derivative:
%	
%	\begin{align*}
%	\frac{\partial^2 
%		E\left( \text{\# Net Change in Supporters} \mid \text{k mentions} \right)}
%	{\partial k^2}
%		& =  (n - r) \log^2 \left( \frac1{1-p} \right) (1 - p)^{k + 1}
%		- s \log^2 \left( \frac1{1-q} \right) (1 - q)^{k + 1} \\
%	\end{align*}
%	
%	And evaluate it at the local extremum:
%	
%	\begin{align*}
%	\frac{\partial^2 
%		E_k}
%	{\partial k^2}
%		& =  (n - r) \log^2 \left( \frac1{1-p} \right) 
%			(1 - p)^{
%				\frac{ \log \left( 
%				\frac{ s \log \left(\frac1{1 - q}\right)}
%				{(n - r) \log \left(\frac1{1-p}\right)} \right)}
%				{ \log \left( \frac{1 - p}{1 - q} \right)}}
%		- s \log^2 \left( \frac1{1-q} \right) (1 - q)^{k + 1} \\
%	\end{align*}
%	
%	
%	If $
%	\frac{\partial^2 
%		E\left( \text{\# Net Change in Supporters} \mid \text{k mentions} \right)}
%	{\partial k^2}$ is negative, then the extremum is a local minimum.
%	Since the function's derivative has a unique zero, this local minimum
%		is the maximum value of th
%	

\subsection{Monte Carlo Simulation}

I conducted a monte carlo simulation, which yielded the attached plots.
The mean was slightly below the theoretical value, while the standard deviation 
	was around 2.6

\subsection{Variance}

The expected number of new supporters given $k$ mentions is the sum of everybody who
	does succumb to pressure, but is also the total number minus the expected
	number of people who don't succumb to pressure.
Recall that $\mathcal{N}(n p, n p (1 - p))$ approximates the sum of $n$ Bernoulli
	random variables with probability $p$.

If there are $k$ mentions of the issue, and a convertable issue-supporter is converted 
	with probability
	$p$, then each issue-supporter will not be converted with probability $(1 - p)^k$.
There are $n - r$ issue-supporters who do not support the candidate.
Therefore, in the limit of a large number of supporters, the number of unconverted
	supporters will follow the following probability distribution:

\begin{align}
\text{Unconverted} 
	& = \mathcal{N}\left((n-r)(1-p)^k, (n-r)(1-p)^k(1-(1-p)^k))\right)
\end{align}

Similarly, the number of non-alienated but alienable supporters will follow the
	following probability distribution:

\begin{align}
\text{Unalienated}
	& = \mathcal{N}\left(s(1-q)^k, s(1-q)^k(1-(1-q)^k))\right)
\end{align}
	
The sum of two uncorrelated normal distributions is 
	$\mathcal{N}(\mu_x, \sigma_x^2) + \mathcal{N}(\mu_y, \sigma_y^2)
		= \mathcal{N}(\mu_x + \mu_y, \sigma_x^2 + \sigma_y^2)$.

The number of non-supporters who are convinced is the total number of convincable non-
	supporters minus
	the non-alienated supporters.
Since the total number of supporters is known, it's a ``random'' variable
	of variance zero: $\mathcal{N}((n - r), 0)$,
	and therefore the total number of non-supporters who are convinced 
	(and alienatables who are alienated) is:

\begin{align}
\text{Converted} 
	& = \mathcal{N} \left( (n-r)(1 - (1 - p)^k), (n - r)(1 - p)^k(1 - (1 - p)^k)\right)\\
\text{Alienated}
	& = \mathcal{N} \left( s(1 - (1 - q)^k), s(1 - q)^k(1 - (1 - q)^k)\right)
\end{align}

Therefore, by the same sum formula, the amount of change in supporters is given
	by the following random variable:

\begin{align}
\text{Change in Support}
	& = \mathcal{N} \left( (n-r)(1 - (1 - p)^k) - s(1 - (1 - q)^k),
		(n - r)(1 - p)^k(1 - (1 - p)^k) + s(1 - q)(1 - (1 - q)^k) \right)
\end{align}

The variance of change in support is therefore:

\begin{align}
\sigma_s^2 & = 
		(n - r)(1 - p)^k(1 - (1 - p)^k) + s(1 - q)^k(1 - (1 - q)^k)
\end{align}

\section{Problem 4}

The sum of $n$ independant Bernoulli trials with probability $p$ is approximated
	in the large-n limit by $\mathcal{N}(n p, n p (1 - p))$.
Therefore, the gaussian is a good way of approximating many different bernoullit trials,
	which might arise in a stochastic population model.
In the population model, each time-step, every female produces offspring with
	probability p.
The total new offspring is the sum of all of the females' offspring.

%	To sample the gaussian distribution, if I have a way of obtaining numbers
%		uniformly spaced from zero to one, I can invert the CDF and apply it 
%		to a random value from zero to one.
%	
%	The CDF of a gaussian distribution is given by:
%	
%	\begin{align}
%	f(x) & = \frac12 \left[ 1 + \erf \left( \frac{x - \mu}{\sqrt2 \sigma} \right) \right] \\
%	f^{-1}(x) & = \mu + \sqrt2 \sigma \erf^{-1} \left( 2 x - 1 \right) 
%	\end{align}
%	
%	Therefore, to sample a gaussian distribution with mean $\mu$ and standard deviation $\sigma$,
%		one should apply $f^{-1}$ to a uniform random sampling from zero to one.
%	I found through numerical tests that this was faster than OCTAVE's sampling function.

\paragraph{}
\begin{tabular}{| l | l | l |}
\hline
Initial Population & Mean Bernoulli Runtime & Mean Gaussian Runtime \\
\hline
4 & 0.015499 sec & 0.020110 sec \\
\hline
10 & 0.021467 sec & 0.020323 sec \\
\hline
40 & 0.047958 sec & 0.019802 sec \\
\hline 
100 & 0.103264 sec & 0.019680 sec \\
\hline
\end{tabular}

\paragraph{}
From the runtimes, we see that for smaller runtimes, the bernoulli trials
	are better, but for larger runtimes, the gaussian approximation is better.
As the initial population grows, it becomes more and more difficult to compute
	the bernoulli trials, getting more difficult approximately linearly, 
	but the difficulty of computing the gaussian
	approximation remains constant.

From the plots, we can see that the smaller the initial population size,
	the worse the agreement with the bernoulli trial will be.

\end{document}
