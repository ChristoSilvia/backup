\documentclass{article}

\usepackage{amsmath}
\usepackage{amsfonts}
\usepackage{listings}

\begin{document}

\title{Math Modeling Assignment 8}
\author{Christopher Silvia}
\maketitle

\section{Rate Changes}

Suppose $b_i$ for $1 \leq i \leq n$ represents the expected number of 
	females birthed in a given year by a single $i$-year old female,
	and $s_i$ for $1 \leq i \leq n-1$ represents the expected fraction 
	of the $i$-year-old	population to survive to age $i+1$ the next year
	\footnote{This assumes that there is an age, $n$, which is the
		maximum age and cannot be exceeded.}
Then, if $\vec{p} \in \mathbb{R}^n$ is a vector representing the population
	one year, then for $A$ defined in (\ref{eq-def-of-A}), the
	expected population is given by $A p$.

\begin{align}
A & =  \left( \begin{matrix}
	b_1 & b_2 & \dots & \dots & b_{n-1} & b_n \\
	s_1 & 0   & \dots & \dots & 0 & 0 \\
	0 & s_2 & \ddots & & \vdots & \vdots \\
	\vdots & & \ddots & \ddots & \vdots & \vdots \\
	\vdots & & & s_{n-2} & 0 & 0 \\
	0 & \dots & \dots & 0 & s_{n-1} & 0 \\
\end{matrix} \right) \label{eq-def-of-A}
\end{align}

We are provided with data for natality and mortality rates from the US census,
	and construct an $A$ matrix corresponding to this data.
The $A$ matrix is constructed in part \ref{script-setup-A} of the code appendix.
The ``stationary age distribution'' is the eigenvector corresponding
	to the largest eigenvalue of the matrix.
The others will be exponentially suppressed, and the stationary distribution
	will not depend on initial conditions, as long as there is a single largest
	eigenvalue.

We rescaled the census data by factors ranging from $0.1$ to $1.0$, and plotted
	the resulting stationary age distributions.

\subsection{Steady Population}

As long as the largest eigenvalue is positive, \emph{at large times}, the population
	will not be shrinking.
	\footnote{Initial conditions can always be chosen so that the population is shrinking:
		for example, if the entire population is placed at a large age where 
		reproduction is unlikely.
	At first, the age distribution will decrease, but eventually, the few children who
		the elderly have will reproduce and we will recover the stationary distribution
		\emph{at large times in the future}}
We want to figure out by how much of a factor US natality can decrease without the 
	population shrinking.
We found that the natality can be multiplied by a factor of $0.488666$ to produce
	a population which is stable.
Any value of $r$ above this will produce a population whose size is asymptotically 
	increasing.
This example is section \ref{script-r-script} in the code appendix.
	\footnote{Sascha Hernandez pointed out that the population could initially
		decrease, for any choice of $r$, and that it was necessary to specify
		what it meant for the population to be ``non-decreasing''}

\section{Maximize Working Age}

From looking at the graph of age distributions, we observe that a larger proportion
	of the people are older as $r$ is decreased, and a larger proportion are younger
	as $r$ is increased.
Suppose we want to maximize the number of working-age adults, from ages 15 to 64.
Which values of $r$ should we choose?

As shown in Plot 2, to achieve a population which is 60\% working age,
	$r$ must be between $0.533371$ and $0.71044$.
The code for plot 2 is given in section \ref{script-maximize-workers} of the code appendix.
The reason that there is such a narrow interval is because if $r$ is too small,
	the population skews too old, while if $r$ is too large, the population
	skews too young.

\section{Code Appendix}

\lstinputlisting[language=Matlab, 
	caption={Code to plot the asymptotic population distributions for values of $r$ from $0.1$ to $1.0$},
	label=script-plot-r]
	{code-for-plotting.m}

\lstinputlisting[language=Matlab, 
	caption={Code to find a value for $r$ such that the largest eigenvalue of $A$ is 1},
	label=script-r-script]
	{code-for-r.m}

\lstinputlisting[language=Matlab,
	caption={Code to find the values of $r$ for which the working age (15-64) population is greater 
		than 60\% of the total population},
	label=script-maximize-workers]
	{code-for-age-windowing.m}

\lstinputlisting[language=Matlab,
	caption={Code to setup the $A$ matrix, with inline US census data},
	label=script-setup-A]
	{setup_a.m}

\end{document}
