\documentclass{article}

\usepackage{amsmath}
\usepackage{amssymb}

\begin{document}

\begin{flushright}
	Chris Silvia
\end{flushright}

\section{Interactions}

\subsection{Commutation Relations}

At any given time these commutation relations are satisfied.
I can justify this by saying that the choice of a specific time
	to be zero is arbitrary, and so for each time $t$, unless
	the lagrangian is time dependent, we can choose to be zero,
	where we know that the commutation relations are valid.
Therefore even in an interacting theory, at every given time
	the commutation relations hold.

\subsection{Interacting Field}

At any fixed time, the operators which create momentum and position eigenstates
	are complete and can span the space of position eigenstates, so when
	we integrate over them with our lorentz-invariant measure 
	$\frac{d^p}{(2\pi)^3} \frac{1}{\sqrt{2 \omega_p}}$ we can express all of the 
	states at any given time.

\subsection{Arbitrary Time}

I'm not sure if this is true.
The operators $a_p$ are evolving with additional dynamics, not just the phase
	changes in the free theory which are incorporated into the lorentz-invariant
	expression for $\phi$.
Therefore, there's no reason to believe that this is an expansion for the field
	at arbitrary times in the future.
I think that the operators need to be brought to the present somehow.
This ``bringing to the present'' will necessarily change the form of the $a_p$ and
	$a_p^\dagger$ operators, because a momentum eigenstate at $t=t_f$ will not
	necessarily also be a momentum eigenstate at $t = 0$.

\section{Time Ordering}

The S-matrix computation is:

$$ \left< f | S | i \right> = \sqrt{2^n \omega_1 \dots \omega_n}
	\left< \Omega | a_{p_3}(\inf) \dots a_{p_n}(\inf) a_{p_1}^\dagger(-\inf)
		a_{p_2}^\dagger(-\inf) | \Omega \right> $$

This is already in time order.
However, we can transform it into something equivalent.
Consider:

$$ \left< \Omega | T\{ \left( a_{p_3}(\inf) - a_{p_3}(-\inf) \right)
		\dots \left( a_{p_n}(\inf) - a_{p_n}(-\inf) \right)
	\left( a_{p_2}^\dagger(\inf) - a_{p_2}^\dagger(-\inf) \right)
	\left( a_{p_1}^\dagger(\inf) - a_{p_1}^\dagger(-\inf) \right) \}
	| \Omega \right> $$

Since there are $n$ two-part terms, there are $2^n$ expanded products.
Then, we apply the time-ordering operator, shifting the $(-\inf)$ terms
	to the right and the $\inf$ terms to the left.

Assume that none of the momenta are the same.
This allows us to freely move the annihilation operators evaluated at the
	same time around in the product.

Now, if a term has an particle annihilation operator $a_{p_i}(-\inf)$
	on the right, it annihilates with $| \Omega \rangle$.
Therefore, any of the $2^n$ product terms which survive have no annihilation
	operators on the right.
Similarly, if there are any creation operators at $a_{p_j}^\dagger(\inf)$,
	then annihilate with $\langle \Omega |$
Therefore, the final product must have \emph{only} creation operators on
	the right, in the $-\inf$ position, and \emph{only} annihilation operators
	on the left, in the $\inf$ position.
Therefore, of all of the $2^n$ terms in the product, only a single one survives:

$$ \left< \Omega | a_{p_3}(\inf) \dots a_{p_n}(\inf) a_{p_1}^\dagger(-\inf)
		a_{p_2}^\dagger(-inf) | \Omega \right> $$

Since Schwartz already proved the following:

$$ \sqrt{2 \omega_p} \left[ a_p(\inf) - a_p(-\inf) \right]
	= i \int d^4x e^{i p x} ( \partial_\mu \partial^\mu + m^2) \phi(x) $$

and the conjugate

$$ \sqrt{2 \omega_p} \left[ a_p^\dagger(\inf) - a_p^\dagger(-\inf) \right]
	= - i \int d^4x e^{-i p x} ( \partial_\mu \partial^\mu + m^2) \phi(x) $$

We can use this expression to find some of the S-matrix elements.

We can bring some of the leading terms inside the time ordering operations:

$$ \left< \Omega | T\{ 
			\left( \sqrt{2 \omega_{p_3}} 
				\left[a_{p_3}(\inf) - a_{p_3}(-\inf) \right] \right)
			\dots
			\left( \sqrt{2 \omega_{p_n}} 
				\left[a_{p_n}(\inf) - a_{p_n}(-\inf) \right] \right)
			\left( \sqrt{2 \omega_{p_2}} 
				\left[a_{p_2}^\dagger(\inf) - a_{p_2}^\dagger(-\inf) \right] \right)
			\left( \sqrt{2 \omega_{p_2}} 
				\left[a_{p_2}^\dagger(\inf) - a_{p_2}^\dagger(-\inf) \right] \right)
		\}
	| \Omega \right> $$

Note that the integrals can be taken out of the time-ordering.
Then we can apply the above formula to the expression, to obtain the LSZ reduction formula:

$$ \langle p_3 \dots p_n | S | p_1 p_2 \rangle>
	= \left[ i \int d^4x_1 e^{- i p_1 x_1}(\partial_\mu \partial^\mu_1 + m^2) \right]
	\dots \left[ i \int d^4x_n e^{ i p_1 x_1}(\partial_\mu \partial^\mu_n + m^2) \right]
	\langle \Omega | T\{\phi(x_1) \dots \phi(x_n) \} | \Omega \rangle $$


\section{Feynmann Propagator}

The feynmann propagator is non-zero for $x_1^0 > x_2^0$ because of the pole in the upper 
	half-plane at $-\omega_k + i \epsilon$, and non-zero for $x_1^0 < x_2^0$ because of 
	the pole in the lower half-plane at $\omega_k - i \epsilon$.
When $x_1^0 > x_2^0$, the upper integration contour is used, and the lower integration contour
	is used in the other case.

\section{Least Clear}

The reading seemed to miss the punch line.
The reading vaguely alluded to the ``goodness'' of the pole in the LSZ operator
	canceling with the pole in the feynmann propagator, but it didn't make much sense.
The reading doesn't connect the whole thing to the differential cross section at the end,
	which is why I thought we were doing this whole thing.
The total cross section is still gotten from integration over $d\Pi_{\text{LIPS}}$,
	which is how it makes sense for $\mathcal{M}$ to be a function of $p_1 \dots p_n$,
	because those are integrated over the whole LIPS to get the total cross section $\sigma$.

The feynmann propagator is calculated in the free theory, but it's only calculated between
	two fields.
We have so far only dealt with $2 \to n$ scattering, so I'm not sure what the interpretation
	of $\langle 0 | T \{ \phi_0(x_1) \phi_0(x_2) \} | 0 \rangle $ is.
Is it $1 \to 1$ scattering?
Does it just represent the amplitude of a particle going from $x_1^\mu$ to $x_2^\mu$ in the
	free theory?
It's somewhat confusing to have this calculation in the free theory right after a discussion
	of scattering.

I assume that we will later compute the time-ordered products in the interacting theory.
But so far, we haven't really connected what we're doing with the main observable, $\sigma$,
	much less figured out how to compute it.

Also, we still haven't discussed the $| \Omega \rangle$, besides the fact that you can hit 
	it with annihilation operators at asymptotic times.

\end{document}
