\documentclass{article}

\usepackage{amsmath}
\usepackage{amsfonts}

\begin{document}

\begin{flushright}
Chris Silvia

Phys 7651

9/13/2015
\end{flushright}

\section{a}

Suppose an accelerator delivers a luminosity of $100 fb^{-1}$
	over the past year.
In that year, they saw $2500$ events which produced and $X$ particle.

Luminosity is defined as:

\begin{align}
dN & = L \times d\sigma
\end{align}

The total number of events is given by $\int dN$.
Therefore, the total cross-section is:

$$ \sigma = \int d\sigma = \int \frac{dN}{L} = \frac{2500}{100 fb^{-1}} = 25 fb $$

For reference, a femptobarn is $1 \times 10^{-43} m^2$.

\section{b}

How do we know that $\delta^{(4)}( \sum p_i^\mu - \sum p_j^\mu )$ is lorentz-
	invariant?

Consider an arbitrary lorentz transformation $\Lambda^\nu_\mu$.
Suppose we shift to a different frame, transforming all of the variables
	in the above expression.

\begin{align*}
\delta^{(4)}( \sum p_i^\mu - \sum p_j^\mu ) & \to  
	\delta^{(4)}\left( \sum \Lambda_\mu^\nu p_i^\mu - \sum \Lambda_\mu^\nu p_j^\mu \right) \\
& \to  \delta^{(4)}\left( \Lambda_\mu^\nu \left( \sum p_i^\mu - \sum p_j^\mu \right) \right) 
\end{align*}

Since the matrix $\Lambda_\mu^\nu$ is a lorentz transformation,
	it is invertible.
Therefore $  \Lambda_\mu^\nu ( \sum p_i^\mu - \sum p_j^\mu ) = 0$
	implies $ \sum p_i^\mu - \sum p_j^\mu = 0$, since $\Lambda_\mu^\nu$
	has no kernel.
Similarly, $  \Lambda_\mu^\nu ( \sum p_i^\mu - \sum p_j^\mu ) \neq 0$
	implies $ \sum p_i^\mu - \sum p_j^\mu \neq 0$.
The functions agree pointwise.

Therefore, since for all lorentz transformations $\Lambda_\mu^\nu$,
$$\delta^{(4)}( \sum p_i^\mu - \sum p_j^\mu ) = 
	\delta^{(4)}( \Lambda_\mu^\nu ( \sum p_i^\mu - \sum p_j^\mu )) $$
	$\delta^{(4)}( \sum p_i^\mu - \sum p_j^\mu )$
	is lorentz-invariant.

\section{c}

The differential cross section for $2 \to n$ particle scattering
	is given by (Schwartz 5.22):

$$ d\sigma =
	\frac{1}{(2 E_1) (2 E_2) \mid \vec{v_1} - \vec{v_2} \mid}
	\mid \mathcal{M} \mid^2
	d\Pi_{\text{LIPS}}
$$

where $d\Pi_{\text{LIPS}}$ is given by:

$$ d\Pi_{\text{LIPS}} := (2 \pi)^4 \delta^{(4)}(\Sigma p) 
	\Pi_j  \frac{d^3 p_j}{(2 \pi)^3} 
		\frac{1}{2 E_j}
$$

I'm trying to answer the question, what are the units of $\mid \mathcal{M} \mid$?
The units of $d\sigma$ should be area, or $-2$.
The units of $\frac{1}{(2 E_1) (2 E_2) \mid \vec{v_1} - \vec{v_2} \mid}$
	are $-2$, since the velocity has energy dimension $0$.
Therefore, we expect the energy dimension of $\mid \mathcal{M} \mid^2$
	to cancel out the energy dimension of $d\Pi_{\text{LIPS}}$.

Using the relation $\delta^4(0) = \frac{T V}{(2 \pi)^4}$,
	we can see that the deltafunction has energy dimension $-4$.
The energy dimension of momentum and energy are both $1$,
	so the energy dimension of $d\Pi_{\text{LIPS}}$
	is $2 n - 4$, since an energy dimension of $2$
	is added for each particle in the product.

Therefore, the energy dimension of $\mid \mathcal{M} \mid^2$
	is $4 - 2 n$.
For 2 particles scattering to 2 particles, the scattering amplitude is unitless.




\end{document}
