\documentclass{article}

\usepackage{amsfonts}

\begin{document}

\begin{flushright}

Chris Silvia

Reading 9-23

\end{flushright}

\section{a}

The denominator in Eq. (7.64) removes terms which have so-called ``bubbles'',
	which are connected subgraphs of the feynmann diagram.
The bubbles do not interact with the main part of the propagator,
	which in a two-point propagator connects $x_1$ to $y_1$.
They instead contribute to the normalization, which we factor out.

I don't know why the bubbles don't need to be removed in the first section.
Since we're interested in calculating correlators, which need to be evaluated
	at endpoints, then the only part of the diagram which can contribute
	to the endpoints is one which is connected to them?

\section{b}

Since the field is given by

\[ \phi(x) = \int \frac{d^3 p}{(2 \pi)^3} \frac{1}{\sqrt{2 \omega_p}}
	\left( a_p e^{-i p_\mu x^\mu} + a_p^\dagger e^{i p_\mu x^\mu} \right) \],

when we are in momentum space, a derivative $\partial_\mu$ gives $i p_\mu$ 
	if a particle is being destroyed, and $- i p_\mu$ if a particle is being created.

From the momentum-space feynmann rules, momentum is conserved at each vertex.

Consider an extra term in the lagrangian, 
	$\mathcal{L}_d[\phi] = \partial_\mu \left( F^\mu(\phi_1 \dots \phi_n) \right)$
	for some $F^\mu$.
This is equal to $\sum_{i=1}^n \frac{\partial F^\mu}{\partial \phi_i}
		\partial_\mu \phi_i$.
As stated before, in momentum space the derivatives of the fields become
	factors of the overall momentum
For this function $F$ to be translation-invariant, it must not be different
	in its arguments.
Therefore, we can factor out $\frac{\partial F^\mu}{\partial \phi_i}$,
	and replace it with:

\[ \mathcal{L}_d [\phi] = \frac{\partial F^\mu}{\partial \phi} i
	\left( {\sum p_\mu}_{\textrm{incoming}} - {\sum p_\mu}_{\textrm{outgoing}} \right) \]

Note that this result only corresponds to the inherently perturbative
	feynmann rules, and says nothing about the exact interactions.

\section{c}

I'm going to give a nod to ``By the way, this integral is infinite...''

It was somewhat surprising that we brushed over the difference
	between $\mid \Omega \rangle$ and $\mid 0 \rangle$ in the 
	lagrangian derivation of the position-space feynmann rules.

It was also somewhat confusing for them to brush over interactions
	with derivatives until later, where they were treated as a special case.


\end{document}
