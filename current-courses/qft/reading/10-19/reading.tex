\documentclass{article}

\usepackage{amsmath}

\begin{document}

\section{Connection between spin and statistics}

The spin-statistics theorem connects the spins of particles,
	and the result of interchanging the particles in a state.
One of the primary results is that particles with integer
	spin are symmetric under interchange of the particles,
	while particles with half-integer spin are antisymmetric.
Among other things, this shows that no two fermions can be
	in the same state.

Spin is connected to statistics by many factors:
	the S-matrix is only lorentz-invariant if the correct 
	statistics are applied.
The assumption that the total energy of the system remain
	bounded from below also implies the spin-statistics
	theorem.
But the main idea is that spin affects how a state
	transforms under spatial translations and rotations.
The half-integer representations mandate antisymmetry
	under exchange, while the integer representations
	mandate symmetry.

\section{Arguments for Spin-Statistics connection}

\begin{itemize}
\item For a field of spin $s$, a rotation of angle $\phi$ induces a 
	phase of $e^{i \phi s}$.
To interchange two identical particles, one way of doing this is
	to rotate the world by $\pi$ around the midpoint of two particles.
Therefore, for integer-spin particles, the phase is $+ 1$ and for
	half-integer spin particles the phase is $- 1$.
This is not a proof.
It merely gives me a good idea of what to try to prove.
If the exchange of particles is not path-dependant, then the spin-statistics
	connection has been determined.
To show that there is no path-dependance, some topologists
	showed that a relative rotation is characterized up to homeomorphism
	by a single angle $\phi$, that which that angles rotated around each other.
In three dimensions, if the particles rotate around each other by $2 \pi$,
	one making a loop inside the other's larger loop, then their loops can
	be unentangled in the third spatial dimension.
So the loops are only defined up to a phase of $\pi$.
Therefore, every exchange of particles is homemorphic to every other, 
	and thus the above argument works.

This argument appeals to my intuition the best, although
	there's a bit of topology magic.
\item Lorentz invariance of the $S$-matrix.
In this case, the commutation relations within the time-ordered product are
	specified, and the lorentz-invariance is checked.
This is convincing but offers no intuition.
\item Stability
I'm confused why the free complex scalar field is discussed before a real
	scalar field.
The argument is clear here.
I don't want this to be the method I use to prove things, though,
	because I would not expect every theory to have a positive definite
	energy density.
\item Causality.
This argument shows that the commutation relations must be correct, otherwise
	the operators of observables would commute (or anticommute) outside
	of the light cone.
\end{itemize}

\section{Confusing}

The arguments about proving the spin-statistics theorem with causality were
	definitely the most confusing in the chapter.
It may just be because of glossing over the definitions of $D$ and $D_1$.

\end{document}
