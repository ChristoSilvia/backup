\documentclass{article}

\usepackage{amsmath}

\begin{document}

\paragraph{Going from spin-0 to spin-1}
\begin{enumerate}
\item To make a relativistic quantum field theory, the field operator
	needs to be lorentz invariant.
If we want our theory to have a single field with multiple degrees of 	
	freedom, it cannot be a lorentz scalar.
The next lorentz-covariant object is a lorentz vector or one-form.
If our space is of dimension $n$, then lorentz-invariant objects
	can only have $n^i$ degrees of freedom, for some $i$, because
	each ``index'' needs four degrees of freedom.
Therefore, if we want objects with a certain number of degrees of freedom,
	we will have to impose additional constraints.
\item A lagrangian must be a lorentz scalar.
How many lorentz scalars can we make with a single vector field $A^\mu$?
	\begin{enumerate}
	\item $A_\mu A^\mu$
	\item $\partial_\mu A^\mu$
	\item $A_\mu \partial_\nu \partial^\nu A^\mu$
	\item $A_\mu \partial_\nu \partial^\mu A^\nu$
	\item Terms with more than two derivatives.
	\end{enumerate}
To see this, we see that any zero-derivative lorentz-invariant quantity
	must only be a function of $A_\mu A^\mu$.
Any one-derivative lorentz-invariant quantity must have the derivative
	contract with something.
Since there's only one field, it has to contract with that.

Any two derivative quantity can have the two derivatives contracted
	with the same index or bearing different indices.
If they bear different indices, then eventually integration by parts
	will push one of the fields contracting one of the indices to the
	left of the derivatives.
The same thing will happen with the same-index case.
Therefore, the following quantities are the only things that lorentz
	scalars can be built from.
This is restrictive.

\item Lagrangians do not necessarily describe physical theories with 
	positive energy densities.
Even lagrangians which containt non-scalars as their fields don't
	necessarily produce theories which behave as non-scalar theories.

Consider the following lagrangian:

\[ \mathcal{L} = \frac{1}{2} A_\mu \partial_\nu \partial^\nu A^\mu 
	+ \frac{1}{2} m^2 A_\mu A^\mu \]

When varied, its equations of motion are:

\[ \left( m^2 + \partial_\nu \partial^\nu \right) A_\mu = 0 \]

A derivative fixes the constraint to be:

\[ \left( m^2 + (a - b) \partial_nu \partial^\nu \right) (\partial_\mu A^\mu) = 0 \]

Therefore, if we choose $a = b$, then this constraint says
	$\partial_\mu A^\mu = 0$ \emph{because $m \neq 0$}.
This constraint is lorentz-invariant and removes one of the four
	degrees of freedom.
Therefore our field has three degrees of freedom as desired.

%	
%	The components of the four-vector are each their own, independant
%		fields.
%	They don't interact, and they should be treated as four seperate
%		fields which each satisfy a seperate klein-gordon equation.
%	The fields transform together, but other than that, they don't
%		interact.
%	
%	We can do some more analysis on this theory to show that it may not
%		even have a positive energy density.
%	This is shown in Schwartz 8.19.
%	
%	We need to chose lagrangians which do not have negative energy densities.
%	
%	\[ \mathcal{L} = \frac{a}{2} A_\mu \partial_\nu \partial^\nu A^\mu
%		+ \frac{b}{2} A_\mu \partial_\mu \partial^\nu A^\nu
%		+ \frac{1}{2} m^2 A_\mu A^\mu \]
%	
%	which is the most general lagrangian with terms of only two or
%		zero derivatives.
%	The equations of motion are (Schwartz 8.21):
%	
%	\[ a \partial_\nu \partial^\nu A_\mu + b \partial_\mu \partial_\nu A^\nu
%		+ m^2 A_\mu = 0 \]
%	

\item We need to determine what the possible polarizations are.
For massive scalar field theories, there are three polarizations.
From the quantum perspective, they correspond to the spin-0,
	spin-1, and spin-(-1) states for a spin-1 particle, which 
	we expect.
\end{enumerate}

\paragraph{From Massive to Massless}
\begin{enumerate}
\item We need to fix a gauge.
Removing the mass term adds an extra symmetry to the lagrangian:

\[ A_\mu(x) \to A_\mu(x) + \partial_\mu \alpha(x) \]

where $\alpha(x)$ is a scalar.

We need to chose a specific gauge for $A_\mu$, which means
	imposing an additional constraint.
One such constraint is the Coulumb gauge.
This constraint removes a degree of freedom from the previously
	mentioned three, so now there are only two.
\item There are only two polarizations.

The interpretation of this is that massless vector fields cannot
	have spin zero, they must have spin -1 or 1.
\end{enumerate}

\end{document}
