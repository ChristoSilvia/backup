\documentclass{article}

\usepackage{amsmath}
\usepackage{amsfonts}

\begin{document}

\section{Why does the Dirac Field need left and right handed components?}

%	Suppose we want to describe a massive, charged particle of spin $\frac12$.
%	Since the particle is charged, we will need the fields to have a gauge
%		transformation.
%	The only two degree of freedom object which represents the Lorentz
%		group is a weyl spinor.
%	A single left or right handed Weyl spinor is already a complex object,
%		and all of its degrees of freedom are spoken for in being a representation
%		of the lorentz group.
%	So, if we want the Weyl spinor to have a nontrivial gauge transformation,
%		and therefore interact with the electromagnetic field,
%		we need to introduce an additional degree of freedom.
%	
%	Since we still want the particle to have spin $\frac12$, the second degree
%		of freedom needs to be another Weyl spinor.
%	So far, we have only determined that the lagrangian must have at least two
%		Weyl spinors.  We haven't determined whether it can have two left-handed
%		Weyl spinors, two right-handed Weyl spinors, or one of each.
%	
%	Suppose we try to write down a kinetic term for a theory with two left-handed
%		Weyl spinors as fields.
%	
%	\begin{align}
%	\mathcal{L}_\text{kinetic} & = i \psi_{L1}^\dagger \bar{\sigma}_\mu 
%			\partial_\mu \psi_{L1}
%		+ i \psi_{L2}^\dagger \bar{\sigma}_\mu \partial_\mu \psi_{L2}
%	\end{align}
%	
%	Therefore, the only kind of gauge invariance that each field could have is:
%	
%	\begin{align}
%	\psi_{L1} & \to e^{i \alpha(x)} \psi_{L1} \\
%	\psi_{L2} & \to e^{i \alpha(x)} \psi_{L2} 
%	\end{align}
%	
%	with the interacting lagrangian being:
%	
%	\begin{align}
%	\mathcal{L}_\text{kinetic} 
%		& = i \psi_{L1}^\dagger \bar{\sigma}_\mu (\partial_\mu + i e A_\mu)\psi_{L1}
%		+ i \psi_{L2}^\dagger \bar{\sigma}_\mu (\partial_\mu + i e A_\mu) \psi_{L2}
%	\end{align}
%	
%	Consider an interaction term between $\psi_L$ and $\psi_R$ of the form
%		$\psi_{L1}^\dagger \psi_{L2}$.
%	
%	Under infinitesimal transformations, the fields $\psi_L$ and $\psi_R$ transform
%		with the following properties:
%	
%	\begin{align}
%	\delta \psi_R & = \frac12(i \theta_j + \beta_j) \sigma_j \psi_R \\
%	\delta \psi_L & = \frac12(i \theta_j - \beta_j) \sigma_j \psi_L \\
%	\delta \psi_R^\dagger & = \frac12( - i \theta_j + \beta_j) \psi_R^\dagger \sigma_j \\
%	\delta \psi_l^\dagger & = \frac12( - i \theta_j - \beta_j) \psi_L^\dagger \sigma_j 
%	\end{align}
%	
%	So let's try $\psi_L^\dagger \psi_L$.
%	
%	\begin{align}
%	\delta( \psi_{L1}^\dagger \psi_{L2}) & =
%		\psi_{L1}^\dagger \frac12 (i \theta_j - \beta_j) \sigma_j \psi_{L2}
%		+ \frac12(-i \theta_j - \beta_j) \psi_{L1}^\dagger \sigma_j \psi_{L2} \nonumber \\
%	& = - \beta_j \psi_{L1}^\dagger \sigma_j \psi_{L2}
%	\end{align}
%	
%	Therefore, $\psi_{L1}^\dagger \psi_{L2}$ is not lorentz-invariant.
%	
%	Since the derivatives happen spatially, with vector indices and not spinor
%		indices, inserting any derivatives into $\psi_{L1}^\dagger \psi_{L2}$
%		will not yield a lorentz-invariant quantity.
%	Therefore, we seem to be stuck.
%	If we chose two fields of the same handedness, for this generalizes to right-handed
%		spinors, then there can be no lorentz invariant interactions.
%	In summary, to have an interacting theory, we need our fields to have gauge
%		transformations.
%%%%%%%%%%%%%%%%%%%%%%%%%%%%%%%%%%%%%%%

Suppose we want to describe a massive particle with spin $\frac12$.
We need a mass term, which will have no derivatives.

Let's assume that our theory has one field, $\psi_L$.
The handedness is irrelevant.
Then, one candidate mass term is $\psi_L^\dagger \psi$.
However, this is not lorentz invariant, which I will now show:

Under infinitesimal transformations, the fields $\psi_L$ and $\psi_R$ transform
	with the following properties:

\begin{align}
\delta \psi_R & = \frac12(i \theta_j + \beta_j) \sigma_j \psi_R \\
\delta \psi_L & = \frac12(i \theta_j - \beta_j) \sigma_j \psi_L \\
\delta \psi_R^\dagger & = \frac12( - i \theta_j + \beta_j) \psi_R^\dagger \sigma_j \\
\delta \psi_L^\dagger & = \frac12( - i \theta_j - \beta_j) \psi_L^\dagger \sigma_j 
\end{align}

So let's try $\psi_L^\dagger \psi_L$.

\begin{align}
\delta( \psi_{L}^\dagger \psi_{L}) & =
	\psi_{L}^\dagger \frac12 (i \theta_j - \beta_j) \sigma_j \psi_{L}
	+ \frac12(-i \theta_j - \beta_j) \psi_{L}^\dagger \sigma_j \psi_{L} \nonumber \\
& = - \beta_j \psi_{L}^\dagger \sigma_j \psi_{L}
\end{align}

Therefore, $\psi_L^\dagger \psi_L$ is not lorentz-invariant, and it cannot
	be used as a mass term.

Let's write $\psi_L 
	= \left( \begin{matrix} \psi_1(x) \\ \psi_2(x) \end{matrix} \right)$.

Mass terms are bilinear in the fields  so therefore, for $A$ an unknown
	matrix with complex entries, a mass term might be $\psi_L^\dagger A \psi_L$.
Then: 
\begin{align}
\delta ( \psi_L^\dagger A \psi_L)
	& = 
	\psi_{L}^\dagger \frac12 (i \theta_j - \beta_j) \sigma_j A \psi_{L}
	+ \frac12(-i \theta_j - \beta_j) \psi_{L}^\dagger A \sigma_j \psi_{L} \nonumber \\
& = \frac i2 \theta_j \psi_L^\dagger [ \sigma_j, A ] \psi_L
- \frac12 \beta_j \psi_{L}^\dagger \{ \sigma_j, A \} \psi_{L}
\end{align}

Therefore, for $\psi_L^\dagger A \psi_L$ is only generically lorentz-invariant
	if $[ \sigma_j, A ] = 0$ and $\{ \sigma_j, A \} = 0$, for each $j$ from 1 to 3.
Let's see what conditions this puts on $A$:

\begin{align*}
[ \sigma_1 , A ] 
	& = \left( \begin{matrix} 0 & 1 \\ 1 & 0 \end{matrix} \right)
	\left( \begin{matrix} a & b \\ c & d \end{matrix} \right)
	- 
	\left( \begin{matrix} a & b \\ c & d \end{matrix} \right)
	\left( \begin{matrix} 0 & 1 \\ 1 & 0 \end{matrix} \right)\\
& = \left( \begin{matrix} c - a & d - b \\ a - c & b - d\end{matrix} \right)\\
[ \sigma_2 , A ]
	& = \left( \begin{matrix} 0 & -i \\ i & 0 \end{matrix} \right)
	\left( \begin{matrix} a & b \\ c & d \end{matrix} \right)
	- 
	\left( \begin{matrix} a & b \\ c & d \end{matrix} \right)
	\left( \begin{matrix} 0 & -i \\ i & 0 \end{matrix} \right)\\
& = i \left( \begin{matrix} - c - b & a - d \\ a - c & b + c \end{matrix} \right)\\
[ \sigma_3, A ]
	& = \left( \begin{matrix} 1 & 0 \\ 0 & -1 \end{matrix} \right)
	\left( \begin{matrix} a & b \\ c & d \end{matrix} \right)
	- 
	\left( \begin{matrix} a & b \\ c & d \end{matrix} \right)
	\left( \begin{matrix} 1 & 0 \\ 0 & -1 \end{matrix} \right)\\
& = \left( \begin{matrix} 0 & b + b \\ - c - c & 0 \end{matrix} \right)\\
\end{align*}

Therefore, since each of these commutators must be zero, $b = c = 0$.
Then, the first relation reduces to:

\begin{align*}
[ \sigma_1 , A ] 
& = \left( \begin{matrix} - a & d  \\ a & - d\end{matrix} \right)\\
\end{align*}

Therefore, lorentz invariance of $\psi_L^\dagger A \psi_L$ implies that $A = 0$.

The only other scalar quantity (up to a multiple) that we can make out of $\psi_L$ is
	$\psi_L^T A \psi_L$.
This transforms as:

\begin{align}
\delta \psi_L^T & = \frac12 (i \theta_j - \beta_j) \psi_L^T \sigma_j^T\\
\delta \psi_R^T & = \frac12 (i \theta_j + \beta_j) \psi_R^T \sigma_j^T
\end{align}

Therefore,

\begin{align}
\delta (\psi_L^T A \psi_L)
	& = \frac12 (i \theta_j - \beta_j) \psi_L^T \sigma_j^T A \psi_L
	+ \psi_L^T A \frac12 ( i \theta_j - \beta_j) \sigma_j \psi_L \nonumber \\
	& = \frac12(i \theta_j - \beta_j) \psi_L^T (\sigma_j^T A + A \sigma_j) \psi_L
\end{align}

If $\psi_L^T A \psi_L$ is lorentz-invariant, then for $j$ ranging from 1 to 3,
	$\sigma_j^T A + A \sigma_j$ should be zero.

\begin{align*}
\sigma_1^T A + A \sigma_1 & = 
	\left( \begin{matrix} 0 & 1 \\ 1 & 0 \end{matrix} \right)
	\left( \begin{matrix} a & b \\ c & d \end{matrix} \right)
	+ 
	\left( \begin{matrix} a & b \\ c & d \end{matrix} \right)
	\left( \begin{matrix} 0 & 1 \\ 1 & 0 \end{matrix} \right)\\
& = \left( \begin{matrix} c + b & d + a \\ a + d & b + c \end{matrix} \right)\\
\sigma_2^T A - A \sigma_2 & = 
	\left( \begin{matrix} 0 & i \\ -i & 0 \end{matrix} \right)
	\left( \begin{matrix} a & b \\ c & d \end{matrix} \right)
	+ 
	\left( \begin{matrix} a & b \\ c & d \end{matrix} \right)
	\left( \begin{matrix} 0 & -i \\ i & 0 \end{matrix} \right)\\
& = i \left( \begin{matrix} c + b & d - a \\ -a + d & - b - c \end{matrix} \right)\\
\sigma_3^T A - A \sigma_3 & = 
	\left( \begin{matrix} 1 & 0 \\ 0 & -1 \end{matrix} \right)
	\left( \begin{matrix} a & b \\ c & d \end{matrix} \right)
	+ 
	\left( \begin{matrix} a & b \\ c & d \end{matrix} \right)
	\left( \begin{matrix} 1 & 0 \\ 0 & -1 \end{matrix} \right)\\
& = \left( \begin{matrix} 2 a & 0 \\ 0 & 2 d \end{matrix} \right)\\
\end{align*}

From the third equation, $a = d = 0$.
From the first equation, $c + b = 0$, so therefore $c = -b$.
The second equation adds no new constraints.
Therefore, if $\psi_L^T A \psi_L$ is lorentz-invariant, then for some
	arbitrary complex $\alpha$, $A$ is of the form:

\begin{align}
A & = \alpha \left( \begin{matrix} 0 & -1 \\1 & 0 \end{matrix} \right)
\end{align}

This is the form of $\sigma_2$, so therefore, any mass term of $\psi_L$
	is proportional to $\psi_L^T \sigma_2 \psi_L$.
This is known as the \emph{Majorana Mass}.

This is not a triumph: we need to remember that the spinor
	is a two-component vector.
If we write $\psi_L = \left( \begin{matrix} \psi_1 \\ \psi_2 \end{matrix} \right)$,
	then $\psi_L^T \sigma_2 \psi_L$ is just $i(\psi_2 \psi_1 - \psi_1 \psi_2)$.
If $\psi_1$ and $\psi_2$ are regular numbers, then we cannot
	use the majorana mass.
If, however, they anticommute for some reason, or have some other
	more complicated commutation relation, then we can still use the
	majorana mass.
Therefore a massive particle which does not couple to the electromagnetic
	field can freely use only one component.

\subsection{Coupling to the Photon Field}

Suppose we want to describe a massive spin-$\frac12$ particle which	
	couples nontrivially to the electromagnetic field.
Since the coupling could only happen through a term such as:
$A_\mu \psi_L^\dagger \bar{\sigma}_\mu \psi_L$, or perhaps
	$A_\mu \psi_L^\dagger \partial_\mu \psi_L$,
Then, the field needs to have a nontrivial gauge transformation to cotransform
	with the field.
It doesn't have any additional degrees of freedom, and it is already
	complex.
Therefore, a single Weyl spinor cannot couple to the electromagnetic field.

A single Weyl spinor can describe a massless particle through the kinetic
	term $i \psi_L^\dagger \bar{\sigma}_\mu \partial_\mu \psi_L$, 
	or a massive particle through the additional mass term
	$\psi_L^T \sigma_2 \psi_L$.
A single Weyl spinor cannot describe a particle which interacts with
	the electromagnetic field.

\subsection{Two-Component Particle}

Now that we've shown that a single Weyl spinor cannot descibe a massive
	spin-$\frac12$ particle which couples to the electromagnetic field,
Multiple spinors are thus required.
I still need to show that a left-handed spinor and a right-handed spinor
	are the required ones.

\section{Components}

$\psi_L$ has two independant components, because it's the $(\frac12,0)$
	representation of $so(1,3)$.
Therefore it has $2J + 1 = 2 \frac12 + 1 = 2$ independent components.

$\psi$ is in $(\frac12,0) \oplus (0,\frac12)$.
Each one has two independant components, so therefore $\psi$ has four 
	independant components.

Finally, $\bar{\psi} \gamma_5 \psi$ has one component, because $\bar\psi$
	is a row matrix(as a spinor), $\gamma_5$ is a square matrix (as a spinor),
	and $\psi$ is a column matrix(as a spinor).
Therefore their product has one component.

\section{Spinor Transformations}

\begin{align*}
\bar\psi & \to \psi^\dagger \Lambda_s^\dagger\\
\psi & \to \Lambda_s \psi \\
\end{align*}

Note that in 10.86 of schwartz, it is shown that:

\begin{align}
(\gamma^0 \Lambda_s \gamma^0)^\dagger
	& = \Lambda_s^{-1}
\end{align}

Therefore, 

\begin{align*}
\psi^\dagger \gamma^0 \psi 
	& \to \psi^\dagger \Lambda_s^\dagger \gamma^0 \psi
\end{align*}

Since $\gamma^0 \gamma^0 = 2 g^{00} = 1$, and ${\gamma^0}^\dagger = \gamma^0$, 
\begin{align*}
& \to \psi^\dagger \gamma^0 \gamma^0 \Lambda_s^\dagger \gamma^0 \Lambda_s \psi \\
& \to \psi^\dagger \gamma^0 ( \gamma^0 \Lambda_s \gamma^0)^\dagger \Lambda_s \psi \\
& \to \psi^\dagger \gamma^0 \Lambda_s^{-1} \Lambda_s \psi\\
\psi^\dagger \gamma^0 \psi & \to \psi^\dagger \gamma^0 \psi
\end{align*}

Similarly, 

\begin{align*}
\psi^\dagger \gamma^0 \gamma^\mu \psi
& \to \psi^\dagger \Lambda_s^\dagger \gamma^0 \gamma^\mu \Lambda_s \psi\\
& \to \psi^\dagger \gamma^0 \Lambda_s^{-1} \gamma^\mu \Lambda_s^{-1} \psi\\
\psi^\dagger \gamma^0 \gamma^\mu \psi
& \to \psi^\dagger \gamma^0 (\Lambda_V)^{\mu \nu} \gamma^\nu \psi \\
\end{align*}

Therefore $\psi^\dagger \gamma^0 \gamma^\mu \psi$ transforms as a vector.

Similarly:

\begin{align*}
\psi^\dagger \gamma^0 \sigma^{\mu \nu} \psi
& \to \psi^\dagger \Lambda_s^\dagger \gamma^0 
	\left( \frac i2 [ \gamma^\mu, \gamma^\nu ] \right) \Lambda_s \psi\\
& \to \psi^\dagger \gamma^0 \Lambda_s^{-1}
	\left( \frac i2 \gamma^\mu \gamma^\nu - \gamma^\nu \gamma^\mu \right) \Lambda_s \psi\\
& \to \psi^\dagger \Lambda_s^\dagger \gamma^0
	(\Lambda_V)^{\sigma \mu} (\Lambda_V)^{\rho \nu}
	\left( \frac i2 [ \gamma^\mu, \gamma^\nu ] \right)\psi\\
& \to \psi^\dagger \Lambda_s^\dagger \gamma^0
	(\Lambda_V)^{\sigma \mu} (\Lambda_V)^{\rho \nu}
	\sigma^{\mu \nu}\psi
\end{align*}

Therefore it transforms as a lorentz tensor.

Finally:

\begin{align*}
\psi^\dagger \gamma^0 \gamma^5 \psi
& \to \psi^\dagger \Lambda_s^\dagger \gamma^0 \gamma^5 \Lambda_s \psi \\
& \to i \psi^\dagger \gamma^0 \Lambda_s^{-1} 
	\gamma^0 \gamma^1 \gamma^2 \gamma^3 \Lambda_s \psi\\
& \to i \psi^\dagger \gamma^0 \Lambda_s^{-1} \gamma^0 \Lambda_s
	\Lambda_s^{-1} \gamma^1 \Lambda_s
	\Lambda_s^{-1} \gamma^2 \Lambda_s
	\Lambda_s^{-1} \gamma^3 \Lambda_s \psi\\
& \to i \psi^\dagger \gamma^0 
	\Lambda_V^{0 \mu} \gamma^\mu
	\Lambda_V^{1 \nu} \gamma^\nu
	\Lambda_V^{2 \rho} \gamma^\rho
	\Lambda_V^{3 \lambda} \gamma^\lambda \psi\\
& \to i \psi^\dagger \gamma^0 
	\Lambda_V^{0 \mu}
	\Lambda_V^{1 \nu} 
	\Lambda_V^{2 \rho} 
	\Lambda_V^{3 \lambda} 
	\gamma^\mu
	\gamma^\nu
	\gamma^\rho
	\gamma^\lambda \psi\\
\end{align*}

\section{Eigenstates}

\begin{tabular}{|c|c|c|c|}
\hline
Value & C Eigenvalue & P Eigenvalue & T Eigenvalue \\
\hline
$\bar\psi \psi$ & 1 & 1 & 1 \\
\hline
$i \bar\psi \gamma^5 \psi$ & 1 & -1 & -1 \\
\hline
$\bar\psi \gamma^\mu \psi$ & -1 & 1 & -1 \\
\hline
$\bar\psi \gamma^\mu \gamma^5 \psi$ & 1 & -1 &-1  \\
\hline
$\bar\psi \sigma^{\mu \nu} \psi$ & -1 & 1 & -1 \\
\hline
\end{tabular}
\end{document}
